\documentclass[a4paper , 10pt]{article}


\usepackage[utf8]{inputenc}
\usepackage[T1]{fontenc}
\renewcommand{\thesubsubsection}{\Alph{subsubsection})}

\begin{document}
\begin{center}
\begin{huge}
\textit{\textbf{RÉALISATION D'APPLICATION (2020)}}
\end{huge}
\end{center}
\begin{center}
\begin{LARGE}
\textit{\textbf{Itération 4}}
\end{LARGE}
\end{center}
\textit{}
\textit{}
\begin{center}
\begin{flushleft}
\begin{Large}
\textbf{\textit{Groupe 2}}
\end{Large}
\end{flushleft}
\end{center}
\begin{center}
03 D\'ECEMBRE 2018
\end{center}

\section{7.42 TIME OUT}
\subsection{7.42.0}
On appel la fonction dans la classe GameView  dans la methode update
on code la methode dans GameModel
Au lieu d'utiliser le temps on compte le nombre de rooms
\begin{verbatim}

    public void timeOut(){
        if(pastRooms.size()==20) {
            gameView.show("Time ouuuuuuuutt\n");
            interpretCommandString("quit");
        }
    }
    
\end{verbatim}

\end{document}

