\documentclass[a4paper , 10pt]{article}


\usepackage[utf8]{inputenc}
\usepackage[T1]{fontenc}
\renewcommand{\thesubsubsection}{\Alph{subsubsection})}

\begin{document}
\begin{center}
\begin{huge}
\textit{\textbf{RÉALISATION D'APPLICATION (2020)}}
\end{huge}
\end{center}
\begin{center}
\begin{LARGE}
\textit{\textbf{Itération 4}}
\end{LARGE}
\end{center}
\textit{}
\textit{}
\begin{center}
\begin{flushleft}
\begin{Large}
\textbf{\textit{Groupe 2}}
\end{Large}
\end{flushleft}
\end{center}
\begin{center}
03 D\'ECEMBRE 2018
\end{center}

\section{7.42 TIME OUT}
\subsection{7.42.0}
On appel la fonction dans la classe GameView  dans la methode update
on code la methode dans GameModel
Au lieu d'utiliser le temps on compte le nombre de rooms
\begin{verbatim}

    public void timeOut(){
        if(pastRooms.size()==20) {
            gameView.show("Time ouuuuuuuutt\n");
            interpretCommandString("quit");
        }
    }
    
\end{verbatim}
\section{7.43}
Il est relativement facile d'implémenter cette fonctionnalité, en effet, depuis notre fichier \textbf{RoomData.csv}, chaque 
salle de jeu possède l'information de sortie, vous pouvez le voir comme un graph orienté.\\
Par exemple, nous pouvont rendre la salle vestiaire comme une salle couloir à sens unique, vous pouvez acceder 
à la salle vestiaire via la porte de la salle de repos, mais la salle de vestiaire n'a qu'une seul porte qui mène à l'aile droite,
enfin l'aile droite ne possède pas de porte qui permet d'y retourner.  
\begin{verbatim}
    vestiaire,in the change room,img/hallway.gif
    south,aile_droite
    NONE
    aile_droite,in the right wing,img/outside.gif
    west,hall,east,cuisine,south,dock2
    NONE
\end{verbatim}

\section{7.44 beamer}
Pour cette fonctionnalité, j'ai decidé d'implémenter tout d'abord beamer 
\begin{verbatim}
cargo1,in the cargo1,img/courtyard.gif
north,cargo2,south,aile_gauche,east,rest
beamer,there is a beamer on the ground,1.0
\end{verbatim}

\end{document}

